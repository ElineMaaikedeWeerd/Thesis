\chapter*{Abbreviations and Conventions}
\label{chapter:conventions}
\addcontentsline{toc}{chapter}{Abbreviations and Conventions}
\section*{}
\singlespacing

\section*{Abbreviations}
\setlength{\tabcolsep}{14pt}
\begin{center}
\begin{tabular}{l | l}
	\textbf{Notation} & \textbf{Description} \\
	\hline
	BAO & Baryon Acoustic Oscillations \\
	CMB & Cosmic Microwave Background \\
	GR & General Relativity (Relativistic) \\
	LIMD & Local-in-mass-density \\
	LSS & Large scale structure \\
	PNG & Primordial non-Gaussianity \\
	RSD & Redshift-space distortion \\
	SKA & Square Kilometre Array \\
	SNR & Signal-to-noise ratio 
\end{tabular}
\end{center}
\section*{Conventions}

\textbf{Fourier convention}


Our Fourier convention is, 
\begin{align*}
	&f(\x) = \int \frac{\diff^3 k}{(2\pi)^3}\, e^{\i \k \cdot \x} f(\k)\,,\\
	&f(\k) = \int \diff^3x\, e^{- \i \k\cdot\x} f(\x) = \int \frac{\diff^3 k'}{(2\pi)^3} (2\pi)^3 \delta^D(\k-\k') f(\k')
\end{align*}

Convolution (FT of a product), 
\begin{equation*}
	h(\k) = \int \frac{\diff^3k_1}{(2\pi)^3} \frac{\diff^3k_2}{(2\pi)^3} f(\k_1) g(\k_2) (2\pi)^3 \delta^D(\k_1 +\k_2 - \k)
\end{equation*}





