\chapter*{Abbreviations and Conventions}
\label{chapter:conventions}
\addcontentsline{toc}{chapter}{Abbreviations and Conventions}
\section*{}
\singlespacing

\section*{Abbreviations}
\setlength{\tabcolsep}{14pt}
\begin{center}
\begin{tabular}{l | l}
	\textbf{Notation} & \textbf{Description} \\
	\hline
	BAO & Baryon Acoustic Oscillations \\
	CDM & Cold Dark Matter \\
	CMB & Cosmic Microwave Background \\
	GR & General Relativity (Relativistic) \\
	IM & Intensity Mapping \\
	LIMD & Local-in-mass-density \\
	LSS & Large scale structure \\
	N & Newtonian \\
	PNG & Primordial non-Gaussianity \\
	RSD & Redshift-space distortion \\
	SKA & Square Kilometre Array \\
	SNR & Signal-to-noise ratio \\
	SN & Supernova(e)
\end{tabular}
\end{center}
\section*{Conventions}

We assume a flat $\Lambda$CDM model, based on GR and perturbed up to second order. 

For numerical forecasts for future surveys, we use the Planck 2018 best-fit parameters.

Our second-order perturbed metric is, 
\begin{equation*}
	a^{-2} \diff s^2 = - \left[ 1 + 2 \Psi^\on + \Phi^\tw \right] \diff \eta^2 + \left[ 1 - 2 \Psi^\on - \Psi^\tw \right] \diff \x^2\,.
\end{equation*}
(In $\Lambda$CDM, $\Phi^\on = \Psi^\on$.)

We expand perturbed quantities as,
\begin{equation*}
	X = X^\on + \frac{1}{2} X^\tw\,.
\end{equation*}

Our Fourier convention is, 
\begin{align*}
	&f(\x) = \int \frac{\diff^3 k}{(2\pi)^3}\, \ee^{\i \k \cdot \x} f(\k)\,,\\
	&f(\k) = \int \diff^3x\, \ee^{- \i \k\cdot\x} f(\x) = \int \frac{\diff^3 k'}{(2\pi)^3} (2\pi)^3 \delta^D(\k-\k') f(\k')\,.
\end{align*}

And the convolution is defined as, 
\begin{equation*}
	h(\k) = \int \frac{\diff^3k_1}{(2\pi)^3} \frac{\diff^3k_2}{(2\pi)^3} f(\k_1) g(\k_2) (2\pi)^3 \delta^D(\k_1 +\k_2 - \k)\,.
\end{equation*}

We use standard orthonormal spherical harmonics,
\begin{equation*}
	Y_{\ell m}(\theta,\varphi) = \sqrt{\frac{2 \ell + 1}{4 \pi}} \sqrt{\frac{(\ell - m)!}{(\ell + m)!}} P_{\ell m}(\cos \theta) \ee^{\i m \varphi}\,.
\end{equation*}






