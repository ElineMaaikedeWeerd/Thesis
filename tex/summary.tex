% !TEX root = ../thesis.tex

\chapter{Conclusions and Outlook}
\label{chapter:summary}
In this thesis, we have examined the relativistic contributions to the observed galaxy bispectrum. For the first part of this thesis, we focussed on the leading-order $\ord(\cH/k)$ corrections, which are absent in the power spectrum of a single tracer, but present in the bispectrum of single tracers and hence form a unique relativistic signature. In the second part of this thesis, we examined the higher-order relativistic corrections, and included local primordial non-Gaussianity in our theoretical description of the galaxy bispectrum. Local PNG and relativistic corrections have similar signatures ($k$-dependence) on large scales. Since detecting primordial non-Gaussianity is a key aim of next-generation (Stage IV) LSS surveys, it is crucial to include both relativistic effecs and PNG in the theoretical description of the observed bispectrum, in order to avoid misinterpreting measurements.

In Chapter~\ref{chapter:introgen}, we introduced the standard model of cosmology, $\Lambda$CDM, which is assumed throughout this work. We descibed the history of observational cosmology, and introduced two statistical tools that are employed to extract information from galaxy surveys: the power spectrum and the bispectrum, the latter of which is the main focus of this work. We briefly introduced primordial non-Gaussianity of the local type, which corresponds to squeezed shape triangles and is predicted by a wide range of inflationary scenarios. PNG gives rise to scale dependence in the galaxy bias, which we discussed next. This scale dependence comes in at next-to-leading-order in $\cH/k$, and motivates the need for inclusion of relativistic corrections.

In Chapter~\ref{chapter:dipole}, we analyse the $\cH/k$ effects in the bispectrum. We have shown that the relativistic galaxy bispectrum has a leading-order correction, which gives rise to a local dipole with resprect to the observer's line-of-sight. This dipole exists even for the bispectrum of a single tracer, in contrast with the power spectrum. Even on equality scales, at redshift $z=1$, the amplitude of this dipole is about 10\% of that of the monopole. Although we have used Gaussian initial conditions in these predicitons, the dipole will be unaffected by non-Gaussianity at leading order, since these corrections start at $\ord(\cH^2/k^2)$. As such, the dipole of the bispectrum is a unique signature of GR on cosmological scales, and offers a new observational window into modifications of GR.

In Chapter~\ref{chapter:detect}, we continue our consideration of the leading-order corrections first examined in Chapter~\ref{chapter:dipole}. Since the relativistic contribution to the bispectrum couples to short-scale Newtonian terms, the signal is not confined to very large scales, unlike the case of the power spectrum. We confirmed the expectations of detectability of such a signal by showing that the signal-to-noise ratio on this relativistic part is $\ord 10$ for a Stage IV H$\alpha$ spectroscopic survey similar to \emph{Euclid}. We checked that the detectability is not compromised by including the uncertainties on two cosmological growth parameters, $\sigma_8$ and $\gamma$, assuming that other cosmological and nuisance parameters are determined by the Newtonian power spectrum and bispectrum. 

The relativistic SNR is dependent on the $k_\mathrm{max}$ assumed, i.e. on the small scales where perturbation theory breaks down, because of the coupling of the relativistic effects to short-scale Newtonian terms. Further, the SNR depends on the largest available scales too, although not as strongly. Little signal is lost if $k_\mathrm{min}$ is increased.  In contrast, the SNR more strongly depends on accurate modelling of the second-order part of the relativistic correction, and in particular is dependent on two astrophysical parameters that do not appear in the Newtonian approximation: the evolution and magnification bias parameters $b_e$ and $\Q$. A key feature of this work was the inclusion of physically self-consistent derivations of these quantities from the luminosity function, which we did for two luminosity function models relevant for future H$\alpha$ surveys.

HI intensity mapping surveys require significant additions to deal with foreground contamination and complexities of instrumental noise, which dominates over shot noise. The loss of signal due to foreground cleaning and telescope beam effects reduces the SNR for next-generation IM surveys on MeerKAT, SKA1-MID and HIRAX, with marginally detectable SNR forecasts of $\sim 6$ for SKA1 and $\sim 5$ for HIRAX. The PUMA survey, which is still in proposal stage, would deliver a SNR of $\sim 14$ in its Petite or first phase (before the futuristic full survey). 

In Chapter~\ref{chapter:ho}, we summarised past work on the Fourier-space galaxy bispectrum including all orders of relativistic projection effects in the observed bispectrum. This meant going beyond the leading-order relativistic correction that was considered in the previous chapters. We included an overview of how the Fourier-space bispectrum of galaxies is constructed from observed galaxy overdensities, and motivate the inclusion of these higher order GR corrections, as they mimic the signal of PNG on large scales. We also showed that corrections that arise from vector and tensor modes are subdominant compared to scalar contributions, and hence can safely be ignored.

Next, in Chapter~\ref{chapter:multipoles}, we present a full analytic decomposition of the Fourier space bispectrum into invariant multipoles about the observer's line-of-sight. We have found that relativistic corrections generate a hierarchy of odd multipoles, which are absent in the Newtonian picture. They also contribute to even multipoles. The leading power of the relativistic correction in each $\ell$ harmonic is $(\cH/k)^1$ for odd multipoles and $(\cH/k)^2$ for even multipoles. We have found that the co-linear configuration only generates non-zero $m = 0$ multipoles, and vanishes for all other values of $m$. Equilateral configurations are always zero for odd $m$, and always zero for the special case of the dipole. Relative to the Newtonian monopole, we have found that all relativistic multipoles decay with redshift. Depending on the scales of interest, higher-order Newtonian terms can exceed second-order GR terms in amplitude. 

Finally, in Chapter~\ref{chapter:localpng}, we have presented the local relativistic corrections to the tree-level redshift-space bispectrum of galaxies in the presence of local PNG. In summary, there are: relativistic projection corrections to the Newtonian RSD at first and second order, relativistic corrections to the Newtonian bias model in the comoving frame at second order, and second-order relativistic projection corrections to the local PNG carried by Newtonian RSD-- from a coupling of first-order scale-dependent bias to first-order relativistic projection effects, and from the linearly evolved local PNG in second-order velocity and metric potentials. We have also shown the bias in the estimate of non-Gaussian parameter $\fnl$ from using a Newtonian analysis, which for a Stage IV survey at $z = 1$ is about $\Delta \fnl \sim 5$, for the long mode above the equality scale. The precise level of bias is of course sensitive to astrophysical parameters and redshift, but the key result is that next-generation precision demands that relativistic corrections are included in the bispectrum. 

Further work could take multiple different directions. Some lower-hanging fruits would be for example Fisher forecasts using the bispectrum to improve cosmological constraints, e.g. including $\fnl$ and scale-dependent bias. Due to the sensitivity of the bispectrum on evolution and magnification bias $b_e$ and $\Q$, it may offer potential for improved constraints on bias parameters. The bispectrum of multiple tracers is not yet fully understood either. The advantages and limitations of measuring parameters using the power spectrum of multiple tracers is already well understood, but it is unclear in what regimes and for what parameters the multi-tracer bispectrum would be a better suited observable. 

In terms of predictions for detectability, it is necessary to include the window function, which we have neglected to do in our Chapter~\ref{chapter:detect}. The window function can also have an imaginary part, similar to the leading-order relativistic correction to the bispectrum. This will need to be corrected for. The dipole that arises from the imaginary part of the bispectrum vanishes identically for equilateral triangle configurations, which may help disentangle the relativistic dipole from that of the window function. There is also no simulation-based model for the bispectrum RSD damping parameter, which we have taken to be equal to the power spectrum damping parameter. From testing the impact of increasing the bispectrum damping parameter, it leads to only a small change in our results, but a complete analysis would require a consistent model of bispectrum RSD damping.

In common with other work on the Fourier-space bispectrum, our analysis as presented here implicitly uses the plane-parallel approximation, since we have fixed the line-of-sight direction $\n$. This approximation can be avoided, for example by using a Fourier-Bessel analysis of bispectrum multipoles, but this will come at the cost of a significant increase in complexity. Note however that errors from the plane-parallel approximation are mitigated in high redshift surveys such as considered in this thesis.

Further work also needs to include the effects of lensing magnification, which are excluded in the standard Fourier analysis. These have been included in work on the galaxy angular bispectrum, and in a spherical Bessel analysis of the bispectrum. 

We also neglected cross-correlations among redshift bins throughout this thesis. This is justified by exquisite redshift accuracy in future surveys, which allows for sharp-edged redshift bins with little to no overlap between them. Integrated effects however will induce correlations along the line-of-sight direction. Ultimately, integrated effects and cross-bin correlations need to be included, which requires a complete treatment using the angular bispectrum or 3-point correlation function. The inclusion of integrated contributions may also change the impact that corrections due to vector and tensor modes have on the bispectrum, and may mean that these corrections can no longer be neglected.

