\chapter{Conclusions and Outlook}
\label{chapter:summary}

In this thesis, we have examined the relativistic contributions to the galaxy bispectrum. Of particular interest is the fact that the leading-order relativistic corrections give rise to a local dipole with respect to the observer's line-of-sight, and mainly due to Doppler-type observational effects. This dipole is purely imaginary, and absent in the Newtonian picture-- hence, if dectected, it would be a smoking-gun signal for relativistic effects in the bispectrum. Furthermore, such a dipole is potentially more detectable in the bispectrum of a single tracer than in the power spectrum, since the leading-order relativistic effects vanish in the power spectrum of a single tracer and one needs to resort to multi-tracer analysis in order for the $\ord(\cH/k)$ effects to be non-vanishing. 

We have found that this dipole is likely detectable in the next-generation spectroscopic and intensity mapping surveys. This prediction is relatively robust despite our analysis at this stage not yet accounting for primordial non-Gaussianity, since scale-dependent effects that arise due to PNG scale as $\ord((\cH/k)^2)$ and higher. A full theoretical analysis of the bispectrum including all orders of relativistic corrections is presented also. From this we find that the relativistic bispectrum gives rise to both even and odd multipoles up to multipole number $\ell = 8$ and $m = 6$, where the odd multipoles arise due to the relativistic parts only. As expected however, the higher multipoles are more suppressed except on very large scales and will be unobservable in next-generation surveys. 

Finally, we include primordial non-Gaussianity of the local type in our theoretical treatment of the bispectrum. PNG gives rise to scale-dependence in the galaxy bias, complicating the Fourier-space kernels. Neglecting this scale-dependence leads to biasing in the estimation of non-linear parameter $\fnl$-- the bias on which can be as large as 5 in a next-generation spectroscopic survey such as $\textit{Euclid}$.

Throughout this work we have made a number of simplifying assumptions. For improved accuracy, these assumptions would need to be abandoned, which would result in a more general theoretical prediction. We have assumed the plane-parallel approximation throughout this thesis. Wide-angle correlations, which we have neglected, might have a non-negligible effect and will need to be accounted for in a future analysis for more robust theoretical predictions. Another sensible next step would be to consider the non-equal time bispectrum, where one correlates different time slices. 