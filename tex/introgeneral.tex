\chapter{Introduction}
\label{chapter:introgen}

Some introduction! General overview of the current observational state of things to make bispectrum work more relevant.

\section{Overview of cosmology}

\section{Formation of structure}

\section{Statistics of galaxy clustering}
\label{chapter:introbisp}

Desciption of the role of statistics in cosmology

The galaxy bispectrum is the Fourier-space equivalent of the three-point function and is as such the first of the higher-order statistics beyond the power spectrum. Next-generation large scale structure such as Euclid (galaxy) and the SKA (21cm intensity mapping) will rely on a combination of the power spectrum and the bispectrum for high-precision measurements of primordial non-Gaussianity and for improvement of constraints on cosmological parameters. In particular, improvement on constraints on the primordial non-Gaussian parameter $\fnl$ will be crucial for discrimitation between various models of inflation and other theories of the early universe. 

As it stands, our current understanding of the universe is that large scale structure of matter is a result of the growth of small primordial fluctuations which have functioned as seeds for structure growth in an otherwise homogeneous universe. These small fluctuations have been amplified by gravitatational instability, resulting in the formation of the structure that we know as the cosmic web on cosmological scales. Tests of theories describing these primordial fluctuations are statistical in nature for the following reasons. For one, there is no direct observationa access to primordial fluctuations, and additionally, the time-scales required to follow cosmological evolution of systems is much much longer than that over which observations are realistically possible. In essence this means that observations on the past lightcone how different objects at different phases of their evolution, and as a result tests of the evolution of large scale structure must be carried out statistically. 

A goal of theoretical cosmology is to make statistical predictions which depend on the statistical properties of primordial perturbations, which in turn lead to the formation of large scale structures in the universe. In these models, the observable universe is modelled simply as a stochastic realisation of a statistical ensemble of possibilities. The most widely considered models are based on the inflationary paradism, and generically give rise to adiabatic Gaussian initial fluctuations. In this case, the origin of stochasticity lies in quantum fluctuations that were generated in the early universe. 

\section{The two-point function}

A purely Gaussian field is fully described by the two-point correlation function or power spectrum. The two-point correlation function is defined as the joint ensemble average of the density at two different points $\x$ and $\x + \textbf{r}$, i.e. 
\begin{equation}
	\xi(r) = \langle \delta(\x) \delta(\x + \textbf{r}) \rangle,
\end{equation}
which is dependent on the distance $r$ between the two points only, due to statistical homogeneity and isotropy which are assumed throughout. Usually, the density contrast $\delta$ is expressed in terms of its Fourier space components, where our Fourier convention is 

\begin{equation}
	\delta({\x}) = \int \frac{\diff^3k}{(2\pi)^3} \, e^{\i \k \cdot \x } \delta(\x),
\end{equation}
where $\delta(\k)$ are complex random variables. Note that there are generally two Fourier conventions that are used in literature on the galaxy statistics, which lead to a difference of $(2\pi)^3$ in the definition of the power spectrum or two-point function. The other choice of Fourier convention is to reverse where the factor of $(2\pi)^3$ goes in the Fourier transforms, that is, using $f(\x) = \int \diff^3k e^{\i \k \cdot \x} f(\k) $ and $f(\k) = \int \frac{\diff^3 x}{(2\pi)^3} e^{-\i \k \cdot \x} f(\x)$ instead of our convention used here.

Since the density contrast is real, this means that we have
\begin{equation}
	\delta(k) = \delta^*(-\k).
\end{equation}

Similarly, the correlators can also be computed in Fourier space, as follows, 
\begin{equation}
	\langle \delta(\k) \delta(\k') \rangle = \int \diff^3 x \diff^3 r \langle \delta(\x) \delta(\x + \r)\rangle e^{-\i (\k + \k') \cdot \x - \i \k' \cdot \r}.
\end{equation}
This can be rewritten using the definition of the two-point correlation function as 
\begin{equation}
	\langle \delta(\k) \delta(\k') \rangle = \int \diff^3 x \diff^3 r \xi(r) e^{-\i (\k + \k') \cdot \x - \i \k' \cdot \r},
\end{equation}
and, performing one of the intregrals, 
\begin{align}
	\langle \delta(\k) \delta(\k') \rangle &= (2\pi)^3 \delta^D(\k + \k') \int \diff^3 r \xi(r) e^{\i \k \cdot \r} \\
	&\equiv (2\pi)^3 \delta^D(\k + \k') P(k),
\end{align}
where $P(k)$ is by definition the density power spectrum.

\section{The three-point function}

Higher-order correlation functions are defined as the connected part of the joint ensemble average of the density in an arbitrary number of locations. In principle it is possible define any order of correlation function like this, but they will rapidly become more computationally complex and expensive. In the case of a purely Gaussian field, the only non-vanishing connected part is the two-point correlation function. This is a direct consequence of Wick's theorem for Gaussian fields, and has a number of important consequences. Firstly it means that a purely Gaussian, statistically homogeneous and isotropic field is fully described by its two-point correlation function or power spectrum, and secondly it means that the statistical properties of any field, which is not necessarily linear, can be written in terms of combinations of two-point correlation functions -- as long as the field is built from a Gaussian field $\delta$. In a generic form, Wick's theorem can be expressed as 
\begin{align}
	&\langle \delta(\ka) \ldots \delta(\bm{k}_{2p + 1}) \rangle = 0 \nonumber \\
	&\langle \delta(\ka) \ldots \delta(\bm{k}_{2p})\rangle = \sum_{[\text{all distinct pairs}]} \prod_{[p \text{ pairs } (i,j)]} \langle \delta(\bm{k}_i) \delta(\bm{k}_j) \rangle.  
\end{align}

More concretely, this means that for a purely Gaussian field, $\langle \delta(\ka) \delta(\kb) \delta(\kc)\rangle = 0$. However, this changes in the presence of any sources of non-linearity. BLAH. An important consequence of non-linear evolution of structure is that the statistics of odd-number density fields are no longer vanishing. The leading odd-number statistic which will be non-zero in the case of non-linear evolution if the three-point correlation function or the bispectrum in Fourier space, 
\begin{equation}
	\langle \delta(\ka) \delta(\kb) \delta(\kc) \rangle = (2 \pi )^3 \delta^\mathrm{D}(\ka + \kb + \kc) \left[ 2 F_2(\ka,\kb) P_\mathrm{L}(\ka) P_\mathrm{L}(\kb) + \text{ 2 c.p.} \right]
\end{equation}
where $F_2$ is the Fourier space density evolution kernel, $P_\mathrm{L}$ is the linear power spectrum from the previous discussion, and redshift dependence is suppressed for brevity. 

Some blah blah about higher order statistics and their importance in future surveys (higher precision data, non-Gaussianities in the universe and effects that give rise to nonlinearities)

\subsection{The matter bispectrum}

The bispectrum is a non-Gaussian statistic, and as such is especially sensitive to any forms of non-linearity in the universe. It is an essential probe for e.g. primordial non-Gaussianity, though there are also other sources of non-Gaussianities in the universe. Primordial non-Gaussianity, which is often parametrised by the non-linear parameter $\fnl$, is predicted by different types of inflation and other theories of the early universe; meaning that improvement of constraints on $\fnl$ could help discriminate between these theories and help shed light on the very early universe and the seeds of structure formation. 

In this section we will go into more detail as to how various amplitudes and signs of the bispectrum correspond to real-space signatures. The bispectrum in Fourier space forms a closed triangle correlating three different wave-vectors and, unlike the power spectrum or two-point function, is able to correlate different scales. The matter bispectrum is unique from the thus far more well-studied CMB bispectrum in that it is able to form a three-dimensional map of the universe, whereas the cosmic microwave background provides a two-dimensional snapshot of the first light only. It is therefore essential to try and improve the theoretical description of the matter bispectrum if we are to utilise the wealth of information from next-generation high-precision galaxy surveys in as good and as unbiased a manner as possible. 

Where the power spectrum is a measure of probability of, e.g. in the case of the galaxy power spectrum, finding galaxies at distance corresponding to separation of points $r$ from each other, the bispectrum similarly maps this to a probability in a three-dimensional equivalent. That is, it can correspond directly to what we know as the cosmic web, and the galaxy or dark matter distributions therein. The bispectrum has degrees of freedom in both modulus of the wavevector i.e. scales of correlation, as well as the shape of the triangle itself. Different triangle shapes correspond to different real-space bispectrum signatures.

Theoretical blah about the matter bispectrum, definitions, showing how various bispectrum signals translate to real space shapes and modulations of signal

\subsection{Matter bispectrum in observations}

Bispectrum in observation -- can skip over any CMB and just focus on lss

\section{Galaxy bias}


(Very) brief overview of the relevant bias


\section{Primordial non-Gaussianity}

Types of primodial non-Gaussianity in the bispectrum, their signatures on various scales, scale dependence it introduces in the clustering bias etc. 


cum sit enim proprium \\
viro sapienti \\
supra petram ponere \\
sedem fundamenti \\
stultus ego comparor \\
fluvio labenti \\
sub eodem tramite \\
nunquam permanenti 