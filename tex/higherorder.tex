% !TEX root = ../thesis.tex

\chapter{Higher order relativistisc effects in the bispectrum}
\label{chapter:ho}

In this chapter, we summarise previous work on relativistic projection effects in the observed galaxy bispectrum~\cite{Umeh:2016nuh,Jolicoeur:2017nyt,Jolicoeur:2017eyi,Jolicoeur:2018blf} going beyond the $\ord(\cH/k)$ approximation used in the previous chapters. For the bispectrum, similar to the power spectrum, effects from observing on the past lightcone need to be taken into account, as they distord the information which is contained in the underlying distribution of dark matter. These lightcone projection effects themselves can also provide new information. The major difference between the power spectrum and bispectrum analyses however, is that for the bispectrum, projection effects up to second order in perturbation theory are required. 

Previously, the GR effects on the angular bispectrum of galaxies arising from lensing convergence has been computed in~\cite{DiDio:2015bua}, which neglects other ultra-large scale GR corrections to the galaxy overdensity. In~\cite{Kehagias:2015tda}, a separate-universe approximation is used to compute the angular bispectrum of galaxies in the squeezed limit only, but including all GR lightcone effects. 