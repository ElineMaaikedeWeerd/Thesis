% !TEX root = ../thesis.tex

\chapter{Higher order relativistisc effects in the bispectrum}
\label{chapter:ho}

In this chapter, we summarise previous work on relativistic projection effects in the observed galaxy bispectrum~\cite{Umeh:2016nuh,Jolicoeur:2017nyt,Jolicoeur:2017eyi,Jolicoeur:2018blf} going beyond the $\ord(\cH/k)$ approximation used in the previous chapters. For the bispectrum, similar to the power spectrum, effects from observing on the past lightcone need to be taken into account, as they distord the information which is contained in the underlying distribution of dark matter. These lightcone projection effects themselves can also provide new information. The major difference between the power spectrum and bispectrum analyses however, is that for the bispectrum, projection effects up to second order in perturbation theory are required. 

Previously, the GR effects on the angular bispectrum of galaxies arising from lensing convergence has been computed in~\cite{DiDio:2015bua}, which neglects other ultra-large scale GR corrections to the galaxy overdensity. In~\cite{Kehagias:2015tda}, a separate-universe approximation is used to compute the angular bispectrum of galaxies in the squeezed limit only, but including all GR lightcone effects. In Section~\ref{sec:localproj} we discuss the Fourier-space observed galaxy biscpectrum including corrections from local projection effects and relativistic corrections from nonlinear dynamical evolution on large scales, which contribute significantly. At second order in general relativity, scalar perturbations also generate second-order tensor and vector modes~\cite{Mollerach:1997up,Matarrese:1997ay}, and these modes enter the observed galaxy number density contrast at second order~\cite{Bertacca:2014hwa,Bertacca:2014wga,Bertacca:2014dra,Yoo:2014sfa,DiDio:2014lka}. However, the power in these second-order tensor and vector modes is much smaller than the power from the scalar modes, and is neglected in forthcoming chapters-- a brief discussion on the vector and tensor modes can be found in Section~\ref{sec:tensorvector}. 


\section{Local lightcone projection effects}\label{sec:localproj}
\todo[inline]{first three papers}

The observed galaxy bispectrum at tree level involves projection effects at both first and second order. These effects, which arise from observing on our past lightcone, include local contributions from Doppler and gravitational potential terms, integrated contributions such as lensing, and at nonlinear order there are also couplings between almost all of these projection effects. On ultra-large scales, the relativistic contributions mimic the effects of scale-dependent bias and hence they are crucial for a complete theoretical description. 

Since we work in Fourier space, which is common with much of the literature on the galaxy bispectrum, we necessarily neglect terms involving lensing and other line-of-sight integrals, i.e. only local relativistic projection effects are included. Furthermore, another approximation which is a direct consequence of working in Fourier space is the plane-parallel approximation, neglecting wide-angle correlations. These are expected to be significant on what scales?\todo{check}, and need to be included for improved theoretical accuracy. When using an angular harmonic or three-point correlation function analysis, the wide-angle effects would be automatically included, but it poses further computational challenges\todo{cite Bertacca?}. 

To be able to compute the observed tree-level bispectrum including local projection effects, we need the observed galaxy number count contrast $\Delta_g^\tw$. The full expression for $\Delta_g^\tw$~\cite{Bertacca:2014dra,Bertacca:2014wga,Bertacca:2014hwa,Yoo:2014sfa,DiDio:2014lka} is very long and complicated, even when integrated contributions are omitted. Conveniently, when neglecting integrated contributions, a fully general form of $\Delta_g^\on$ and $\Delta_g^\tw$ can be expressed in Poisson gauge. All terms at the observer are neglected too, since they do not contribute to the bispectrum. 

Important to note is how the bispectrum is still dependent on the magnification bias, despite omitting integrated effects. This is because GR weak lensing convergence consists of both the standard integrated term and local non-integrated terms~\cite{Bonvin:2008ni}, and hence magnification bias still enters the bispectrum even when integrated terms are ignored. Previous sections have already illustrated the dependence of bispectrum power and detectability on the evolution and magnification biases, highlighting the importance of modelling these self-consistently from the same luminosity function.\todo{check with previous sections what's said exactly} 

The observed number density contrast $\Delta_g$ is defined through, 
\begin{equation}
	\frac{\diff N(z, \n > \ln L)}{\diff z \diff \Omega_o} = \frac{\chi^2(z)}{(1 + z)^4 \cH(z)} \bar{\N}(z, > \ln L) \left[ 1 + \Delta_g(z, \n, >\ln L) \right],
\end{equation}
where $\diff N$ is the observed count of galaxies above threshold luminosity $L$, in direction of observation $\n$, within redshift interval $\diff z$ and within solid angle element $\diff \Omega_o$. $\cH(\eta) = a'(\eta)/a(\eta)$ is the conformal hubble rate, and $\chi$ is the comoving line-of-sight distance.

\section{Contributions from second-order vector and tensor modes}\label{sec:tensorvector}

These are subdominant and neglected after- include a brief discussion (paper IV\cite{Jolicoeur:2018blf})



