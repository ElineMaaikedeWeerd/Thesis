Here we present a detailed derivation of the analytic result of the integration of
\begin{equation} \label{eq:xabint}
X_{\ell m}^{ab} = \int_0^{2\pi} \diff\varphi\, \int_{-1}^1 \diff\mu_1\, (\i \mu_1)^a (\i \mu_2)^b \,Y_{\ell m}^*(\mu_1,\varphi)\,.
\end{equation}
In the above, we have used that $\mu_1 = \cos \theta$, such that $Y_{\ell m} (\theta, \varphi) = Y_{\ell,m}(\mu_1,\varphi)$. 
The standard orthonormal spherical harmonics are defined as, 
\begin{equation}
	Y_{\ell m}(\mu_1, \varphi) = \sqrt{\frac{2\ell+1}{4\pi}} \sqrt{\frac{(\ell - m)!}{(\ell + m)!}} P_{\ell m}(\mu_1) e^{\i m \varphi}\,,
\end{equation}
and the spherical harmonics are related to their complex conjugate as, 
\begin{equation}
	Y_{\ell,-m} = (-1)^m Y^*_{\ell,m}\,,\quad Y^*_{\ell,m} = (-1)^m Y_{\ell,-m}\,.
\end{equation}
Equation~\eqref{eq:xabint} is separable. We can make this explicit as follows. First expressing $\mu_2$ in terms of $\mu_1$ using $\mu_2 = \sqrt{1 - \mu_1^2} \sin\theta \sin\varphi + \mu_1 \cos\theta$, where $\theta \equiv \theta_{12} \neq \theta_1$-- it is the angle between vectors $\ka$ and $\kb$. Then we can expand $(\i\mu_2)^b$ using the binomial series, 
\begin{align}
	(\i \mu_2)^b &= \i^b ( \sqrt{1 - \mu_1^2} \sin\theta \sin\varphi + \mu_1 \cos\theta)^b \nonumber \\ 
	&= \i^b \sum_{g=0}^b \binom{b}{g} \left[ \sqrt{1 - \mu_1^2} \sin\theta \sin\varphi \right]^g \left[ \mu_1 \cos\theta \right]^{b-g}\nonumber \\
	&= \i^b \sum_{g=0}^b \binom{b}{g} (1 - \mu_1^2)^{g/2} \mu_1^{b-g} \sin^g\varphi \sin^g\theta \cos^{b-g}\theta\,.
\end{align}
Using this, we have,
\begin{align}
	X_{\ell m}^{ab} =& \i^{a+b} \sum_{g=0}^b \sin^g\theta \cos^{b-g}\theta \binom{b}{g} \int_0^{2\pi}\diff\varphi\, \int_{-1}^1\diff\mu (1-\mu^2)^{g/2} \mu^{a+b-g} \times \nonumber \\
	&\quad \sin^g\varphi Y^*_{\ell m}(\mu,\varphi)\,,
\end{align}
where we have dropped subscript on $\mu_1 \equiv \mu$ for brevity. Now using the definition of the complex-conjugated standard spherical harmonics, 
\begin{align}
	Y^*_{\ell m} &= (-1)^m Y_{\ell,-m} \nonumber \\
	&= (-1)^m \sqrt{\frac{2\ell + 1}{4 \pi}} \sqrt{\frac{(\ell +m)!}{(\ell - m)!}} P_{\ell,-m}(\mu) e^{-\i m \varphi} \,
\end{align}
and the associated Legendre polynomials $P_{\ell m}$ can be rewritten for negative $m$ as, 
\begin{equation}
	P_{\ell,-m} = (-1)^m \frac{(\ell - m)!}{(\ell + m)!} P_{\ell,m}\,,
\end{equation}
such that, 
\begin{equation}
	Y^*_{\ell m} = \sqrt{\frac{2\ell + 1}{4 \pi}} \sqrt{\frac{(\ell - m)!}{(\ell + m)!}} P_{\ell m}(\mu) e^{-\i m \varphi} \,.
\end{equation}
This makes the separability of the integral~\eqref{eq:xabint} explicit, 
\begin{align}
	X_{\ell m}^{ab} =& \i^{a+b} \sum_{g=0}^b \binom{b}{g} \sin^g\theta \cos^{b-g}\theta \sqrt{\frac{2\ell + 1}{4\pi}} \sqrt{\frac{(\ell - m)!}{(\ell + m)!}} \int_0^{2\pi}\diff\varphi\, \sin^g\varphi e^{-\i m \varphi} \times \nonumber \\
	& \int_{-1}^1 \diff\mu\, (1-\mu^2)^{g/2} \mu^{a+b-g} P_{\ell m}(\mu)\,.
\end{align}
Now these two integrals can be solved independently. Starting with the integral over $\mu$, splitting the interval $\int_{-1}^1 = \int_{-1}^0 + \int_0^1$, the $\int_{-1}^1$ term can be rewritten using a change of variables and the parity of the associated Legendre polynomials as, 
\begin{align}
	&\int_{-1}^0\diff\mu\, (1-\mu^2)^{g/2} \mu^{a+b-g} P_{\ell m}(\mu) \nonumber \\
	&= - \int_0^{-1}\diff\mu\, (1-\mu^2)^{g/2} \mu^{a+b-g} P_{\ell m}(\mu) \nonumber \\
	&= \int_0^1\diff\tilde{\mu}\, (1-\tilde{\mu}^2)^{g/2}(-\tilde{\mu})^{a+b-g} P_{\ell m}(-\tilde{\mu}) \nonumber \\
	&= (-1)^{\ell + m + a + b - g} \int_0^1 \diff\mu\, (1-\mu)^{g/2} \mu^{a+b-g} P_{\ell m}(\mu)\,,
\end{align}
where we have used that $P_{\ell m}(-x) = (-1)^{\ell + m} P_{\ell m}(x)$, and renamed $\tilde{\mu} = \mu$ in the last line. The full integral over $\mu$ then can be written as, 
\begin{equation}
	\left[ 1 + (-1)^{\ell+m+a+b-g} \right] \int_0^1 \diff\mu\, (1-\mu^2)^{g/2} \mu^{a+b-g} P_{\ell m}(\mu)\,.
\end{equation}
The associated Legendre polynomials can be written in closed form, using, 
\begin{align}
	P_{\ell m}(\mu) = (-1)^m P_\ell^m \nonumber \\
	&= (-1)^m (-1)^m 2^\ell (1-\mu^2)^{m/2} \sum_{k = m}^\ell \frac{k!}{(k-m)!} \mu^{k-m} \binom{\ell}{k} \binom{\frac{1}{2}(\ell + k - 1)}{\ell} \nonumber \\
	& 2^\ell (1-\mu^2)^{m/2} \sum_{k = m}^\ell \frac{k!}{(k-m)!} \mu^{k-m} \binom{\ell}{k} \binom{\frac{1}{2}(\ell + k - 1)}{\ell}\,,
\end{align}
such that, 
\begin{align}
	&\left[ 1 + (-1)^{\ell+m+a+b-g} \right] (-1)^m 2^\ell \sum_{k=m}^\ell \frac{k!}{(k-m)!} \binom{\ell}{k} \binom{\frac{1}{2}(\ell + k - 1)}{\ell} \times \nonumber \\
	&\int_0^1 \diff\mu\, (1-\mu^2)^{\frac{1}{2}(m+g)} \mu^{a+b+k-g-m}\,.
\end{align}
Changing variables $\xi = \mu^2$, omitting the prefactor for brevity, we get, 
\begin{equation}
	\int_0^1\diff\xi\, \frac{1}{2} (1 - \xi)^{\frac{1}{2}(m+g)} \xi^{\frac{1}{2}(a+b+k-g-m-1)}\,.
\end{equation}
Comparing to the Beta function, 
\begin{equation}
	B(x,y) = \int_0^1\diff t\, (1-t)^{y-1} t^{x-1}\,,
\end{equation}
we can identify $x = \frac{1}{2}(a+b+k-g-m+1)$ and $y = \frac{1}{2}(m+g+2)$. The Beta function can be expressed in terms of Gamma functions $B(x,y) = \frac{\Gamma(x)\Gamma(y)}{\Gamma(x+y)}$, which in turn can be written in terms of factorials, for any positive integer $n$ this relation is $\Gamma(n) = (n-1)!$. We will express the result in terms of Gamma functions as, 
\begin{align}
	&\int_{-1}^{1}\diff\mu\, (1-\mu^2)^{g/2} \mu^{a+b-g} P_{\ell m}(\mu) \nonumber \\
	&= \left[ 1 + (-1)^{\ell+m+a+b-g} \right] (-1)^m 2^\ell \sum_{k=m}^\ell \frac{k!}{(k-m)!} \binom{\ell}{k} \binom{\frac{1}{2}(\ell + k - 1)}{\ell} \times \nonumber \\
	& \left\{ \Gamma\left[\frac{1}{2}(m+g+2)\right] \Gamma\left[\frac{1}{2}(a+b+k-g-m+1)\right] \right\}\cdot\left\{ \Gamma\left[\frac{1}{2}(a+b+k+3) \right]\right\}^{-1}\,.
\end{align}
Now to solve the integral over $\varphi$, 
\begin{equation}
	\int_0^{2\pi}\diff\varphi\, \sin^g\varphi e^{-\i m \varphi}\,.
\end{equation}
Write $\sin^g\varphi$ in terms of $e$ using the usual trig identities and binomial expansion as, 
\begin{equation}
	\sin^g\varphi = \frac{1}{(2\i)^g} \sum_{n=0}^g \binom{g}{n} e^{\i (g-n) \varphi} (-1)^n e^{-\i n \varphi}\, 
\end{equation}
to get, 
\begin{align}
	&\int_0^{2\pi}\diff\varphi\, \frac{1}{(2\i)^g} \sum_{n=0}^g \binom{g}{n} (-1)^n e^{\i \varphi (g-m-2n)}\nonumber \\
	&= \frac{1}{(2\i)^g} \sum_{n=0}^g \binom{g}{n} (-1)^n \int_0^{2\pi}\diff\varphi\, e^{\i \varphi (g-m-2n)} \nonumber \\
	&= \frac{1}{(2\i)^g} \sum_{n=0}^g \binom{g}{n} (-1)^n (2\pi) \delta_{g-m-2n, 0}\,.
\end{align}
The Kronecker $\delta$ picks out the term in the sum which satisfies $g - m -2n =0 \Rightarrow n = \frac{1}{2}(g-m)$, s.t. 
\begin{equation}
	\int_0^{2\pi} \diff\varphi\, \sin^g\varphi e^{-\i m \varphi} = \frac{1}{(2\i)^g} \binom{g}{\frac{1}{2}(g-m)} (2\pi) (-1)^{\frac{1}{2}(g-m)}\,
\end{equation}
where $n$ must be a positive integer (or zero), so $g \geq m$ and $g -m$ even.



\iffalse

Here we present the derivation of the analytic result \ref{eq:sumformula}, that is, exact integration of:
\begin{equation} X_{\ell m}^{ab} = \int_0^{2\pi} \diff\varphi\, \int_{-1}^1 \diff\mu_1\, (\i \mu_1)^a (\i \mu_2)^b \,Y_{\ell m}^*(\mu_1,\varphi).
\end{equation}
We will calculate this for $m\geq0$, as for negative $m$ we can use the result
\begin{equation}
X_{\ell, -m}^{ab} = (-1)^{a+b+m} X_{\ell m}^{ab*}\,,
\end{equation}
which follows on using the complex conjugate of the standard orthonormal spherical harmonics,
\begin{align}
Y_{\ell m}^* &= (-1)^m Y_{\ell, -m} \nonumber\\
&= \sqrt{\frac{2 \ell + 1}{4 \pi}} \sqrt{\frac{(\ell - m )!}{(\ell + m )!}} P_\ell^m(\mu) e^{-\i m \varphi}.
\end{align}

To perform this integral analytically, first use the binomial expansion to expand the \(\mu_i\) dependence in the integrand, 
\begin{equation}
( \i \mu_1)^a \left(\i \sqrt{1-\mu_1^2} \sin\theta \sin\varphi + \i \mu_1 \cos\theta \right)^b,
\end{equation}
as
\begin{align}
(\i \mu_2)^b &= \i^b \sum_{g=0}^b \binom{b}{g} (\mu_1 \cos\theta)^{b-g} \left(\sqrt{1-\mu_1^2}\sin\theta \sin\varphi\right)^g \nonumber\\
& =\i^b \sum_{g=0}^b \binom{b}{g} \mu_1^{b-g} \left(\sqrt{1-\mu_1^2}\right)^g \cos\theta^{b-g} \sin\theta^g \sin\varphi^g. 
\end{align}
Now the separability of the angular parts of the integrand has been made explicit. Inserting this expansion backinto the integral we get, 
\begin{align}
&\sum_{g=0}^b \i^{a+b} \cos^{b-g}\theta \sin^g\theta \binom{b}{g} \int_0^{2\pi}\diff\varphi\, \int_{-1}^1 \diff\mu_1\, \mu_1^{a+b-g} \left(1-\mu_1^2\right)^{g/2} \sin^g\varphi Y_{\ell m}^*(\mu_1,\varphi),
\end{align}
where the factors that are independent of integration angles \(\mu_1,\,\varphi\) have been taken out of the integral (note that \(\theta = \theta_{12}\) as per our convention used throughout this paper). 

In what follows will drop the subscript on \(\mu_1 = \mu\) for convenience. Using the standard definition of the spherial harmonics, the integral then becomes,
\begin{equation}
	\sum_{g=0}^b \i^{a+b} \cos^{b-g}\theta \sin^g\theta \binom{b}{g} \sqrt{\frac{2\ell+1}{4 \pi}} \sqrt{\frac{(\ell-m)!}{(\ell+m)!}}\int_0^{2\pi}\diff\varphi\, \int_{-1}^1 \diff\mu\, \mu^{a+b-g} (1-\mu^2)^{g/2} \sin^g\varphi P_\ell^m(\mu) e^{-\i m \varphi},
\end{equation}
and hence can easily be split into two parts. The associated Legendre polynomials \(P_\ell^m\) can be expressed as 
\begin{equation}
P_\ell^m(\mu) = (-1)^m (1-\mu^2)^{m/2} \frac{\diff^m}{\diff\mu^m} P_\ell(\mu), 
\end{equation}
i.e. as full derivatives of the Legendre polynomials. These in turn can be expressed as a sum 
\begin{equation}
P_\ell(\mu) =2^\ell \sum_{h=0}^\ell \mu^\ell \binom{\ell}{h} \binom{\frac{1}{2}(\ell+h-1)}{\ell}.
\end{equation}
Using the Legendre polynomials in this form and substituting in,
\begin{align}
	&\int_{-1}^1 \diff\mu\, \mu^{a+b-g} \left(1-\mu^2\right)^{g/2} P_\ell^m(\mu) \nonumber \\
	&= (-1)^m 2^\ell \sum_{h=0}^\ell \binom{\ell}{h} \binom{\frac{1}{2}(\ell+h-1)}{\ell} \int_{-1}^1 \diff\mu\, \mu^{a+b-g} \left(1-\mu^2\right)^{g/2}  (1-\mu^2)^{m/2} \frac{\diff^m}{\diff\mu^m} \mu^\ell \nonumber \\
	&= (-1)^m 2^\ell \sum_{h=0}^\ell \binom{\ell}{h} \binom{\frac{1}{2}(\ell+h-1)}{\ell} \frac{h!}{(h-m)!} \frac{1}{2} \left(1+ (-1)^{a+b-g+h-m}\right) \Gamma\left[\frac{1}{2}(a+b-g+h-m-1)\right] \Gamma\left[\frac{1}{2} (g+m+2)\right] \nonumber \\
	&\times  \left\{ \Gamma\left[\frac{1}{2} (a+b+h+3)\right] \right\}^{-1},
\end{align}
where \(m \geq h\), so that above result may be written as
\begin{align}
	&(-1)^m 2^\ell \sum_{h=m}^\ell \binom{\ell}{h} \binom{\frac{1}{2}(\ell+h-1)}{\ell} \frac{h!}{(h-m)!} \frac{1}{2} \left(1+ (-1)^{a+b-g+h-m}\right) \Gamma\left[\frac{1}{2}(a+b-g+h-m-1)\right] \Gamma\left[\frac{1}{2} (g+m+2)\right] \nonumber \\
	&\times  \left\{ \Gamma\left[\frac{1}{2} (a+b+h+3)\right] \right\}^{-1}.
\end{align}
Evaluating now the integral over \(\varphi\), which is, 
\begin{align}
	&\int_0^{2\pi} \diff\varphi\, \sin^g \varphi e^{-\i m \varphi} \nonumber \\
	& = \frac{1}{(2 \i)^g} \sum_{n=0}^g \binom{g}{n} (-1)^n \int_0^{2\pi}\diff\varphi\,  e^{\i (g-m-2n )\varphi} \nonumber \\
	&= \frac{1}{(2 \i)^g}  \sum_{n=0}^g \binom{g}{n} (-1)^n (2 \pi) \delta_{g-m-2n, 0}.
\end{align}

The Kronecker \(\delta\) picks out one of the terms in the sum, \(g-m-2n=0 \to n = \frac{g-m}{2}\), so 
\begin{equation}
	\int_0^{2\pi} \diff\varphi\, \sin^g \varphi e^{-\i m \varphi} = \frac{1}{(2 \i )^g} \binom{g}{\frac{1}{2}(g-m)} 2 \pi (-1)^{\frac{g-m}{2}},
\end{equation}
if \(g-m\) is even, in which case \((-1)^{\frac{g-m}{2}} = 1\), so
\begin{equation}
	\int_0^{2\pi} \diff\varphi\, \sin^g \varphi e^{- \i m \varphi} = 2^{-g} \i^g (-1)^g \binom{g}{\frac{1}{2}(g-m)} 2 \pi, 
\end{equation}
for \(g+m \) even, zero otherwise. 

Putting the results from both integrals together, 
\begin{align}
	&\int_0^{2\pi} \diff\varphi\, \int_{-1}^1 \diff\mu_1\, (\i \mu_1)^a (\i \mu_2)^b \,Y_{\ell m}^*(\mu_1,\varphi) = \sum_{g=0}^b \i^{a+b} \cos^{b-g}\theta \sin^g\theta \binom{b}{g} \sqrt{\frac{2\ell+1}{4 \pi}} \sqrt{\frac{(\ell-m)!}{(\ell+m)!}} \times \left\{(-1)^m 2^\ell \sum_{h=m}^\ell \binom{\ell}{h} \right. \nonumber \\
	&\left. \binom{\frac{1}{2}(\ell+h-1)}{\ell} \frac{h!}{(h-m)!}  \frac{1}{2} \left(1+ (-1)^{a+b-g+h-m}\right) \Gamma\left[\frac{1}{2}(a+b-g+h-m-1)\right] \Gamma\left[\frac{1}{2} (g+m+2)\right]  \times \right. \nonumber \\
	&\left.  \left[\Gamma\left[\frac{1}{2} (a+b+h+3)\right] \right\}^{-1} \right]\times \left\{  2^{-g} \i^g (-1)^g \binom{g}{\frac{1}{2}(g-m)} 2 \pi\right\}.
\end{align}
Simplifying the above result,
\begin{align}
	&\sum_{p=m}^{\frac{1}{2}(b+m)} \i^{a+b+m} \cos^{b-2p+m}\theta \sin^{2p-m}\theta \frac{b!}{(2p-m)! (b-2p+m)!} \sqrt{\frac{\pi (2\ell+1) (\ell-m)!}{(\ell+m)!}} \sum_{q=m}^\ell 2^\ell \frac{\ell!}{q! (\ell-q)!} \nonumber \\
	& \frac{(\frac{1}{2}(\ell+q-1))!}{\ell!(\frac{1}{2}(\ell+q-1)-l)!} \frac{q!}{(q-m)!} 2^{-1} (1+(-1)^{a+b+q}) \Gamma\left[\frac{1}{2} (a+b-2p+q+1)\right] \Gamma\left[\frac{1}{2}(2p+2)\right]\times \nonumber \\
	& \frac{(2p-m)!}{(\frac{1}{2}(2p-2m))! (2p-m - \frac{1}{2}(2p-2m))!} \times \left\{ \Gamma\left[\frac{1}{2}(a+b+q+3)\right] \right\}^{-1},
\end{align}
collecting terms, and after cancellations some cancellations obtain, 
\begin{align}
	&\sum_{p=m}^{\frac{1}{2}(b+m)} \sum_{q=m}^\ell 2^{\ell+m-1} \i^{a+b+m} \cos^{b-2p+m}\theta \sin^{2p-m}\theta  \sqrt{\frac{\pi (2\ell+1) (\ell-m)!}{(\ell+m)!}} b! \left(\frac{1}{2}(\ell+q-1)\right)! \left(\frac{1}{2}(a+b-2p+q-1)\right)! \times \nonumber \\
	& \left(1+(-1)^{a+b+g}\right) \times \left[ 4^p (p-m)! (b+m-2p)! (\ell-q)! \left(\frac{1}{2}(-\ell+q-1)\right)! (q-m)! \left(\frac{1}{2}(a+b+q+1)\right)! \right]^{-1},
\end{align}
where we have used \(\Gamma(n)=(n-1)!\) to rewrite the gamma functions in terms of factorials, that this is non-zero only if \(\frac{1}{2}(g-m)\) is even, and that \((-1)^{w}=1\) if \(w\) is even. The final analytic expression for \(m>0\) hence is,  
\begin{align}
\int_0^{2\pi}{\diff}\varphi\,& \int_{-1}^{1}{\diff}\mu_1\,(\i\mu_1)^a\left(\i\sqrt{1-\mu_1^2} \sin\theta \sin\varphi + \i\mu_1\cos\theta\right)^b \, Y^*_{\ell m} (\mu_1,\varphi)  = 2^{\ell+m-1}\i^{a+b+m}\sqrt{\frac{\pi(2\ell+1)(\ell-m)!}{(\ell+m)!}} \nonumber \\
&\times
\sum_{p=m}^{\frac{1}{2}(b+m)}\sum_{q=m}^{\ell}
\frac{\left[1+(-1)^{a+b+q}\right]\,b!\,\cos^{b+m-2p}\theta\sin^{2p-m}\theta}{4^p(b+m-2p)!(\ell-q)!(p-m)!(q-m)!}
\frac{\Gamma\left[\frac{1}{2}(q+\ell+1)\right]}{\Gamma\left[\frac{1}{2}(q-\ell+1)\right]}
\frac{\Gamma\left[\frac{1}{2}(a+b +q-2p+1)\right]}{\Gamma\left[\frac{1}{2}(a+b+q+3)\right]}.
\end{align}
Note that in the above we have kept the expression in terms of gamma functions, but this can easily be reverted back to the factorial notation.

\fi 
