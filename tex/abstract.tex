\chapter*{Abstract}
\label{ch:abstract}
\addcontentsline{toc}{chapter}{Abstract}
\section*{}
\singlespacing

% \thispagestyle{empty} if wanting to remove page numbering from abstract
The bispectrum is the Fourier-space equivalent of the three-point correlation function; the first of the higher-order statistics beyond the power spectrum or two-point correlation. 
The galaxy bispectrum will play an important role in next-generation galaxy surveys. On large scales, it will be a key probe for measuring primordial non-Gaussianity, improve constraints on cosmological parameters, and hence help discriminate between various models of inflation and other theories of the early universe. On these scales a variety of relativistic effects come into play once the galaxy number-count fluctuation is projected onto the past light cone. It is important for these effects to be taken into account in our theoretical treatment of the bispectrum, as they will contaminate the primordial non-Gaussian signal and bias measurements. 

In this thesis we review the relativistic projection effects in the galaxy bispectrum up to all orders of $\cH/k$, and examine how these relativistic effects contribute to the invariant multipoles of the Fourier-space bispectrum about the observer's line of sight. The Fourier-space bispectrum is complex, with an imaginary part arising from the relativistic effects only, which generates odd multipoles of the bispectrum. This means that detection of this imaginary part is a smoking gun signal of the relativistic contributions, and we show that such a signal is detectable in future surveys, although with a higher signal-to-noise ratio for spectroscopic surveys compared to 21cm intensity mapping surveys. Finally, we include local primordial non-Gaussianity in the theoretical description of the relativistic bispectrum. 