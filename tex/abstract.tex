\chapter*{Abstract}
\label{ch:abstract}
\addcontentsline{toc}{chapter}{Abstract}
\section*{}
\singlespacing

% \thispagestyle{empty} if wanting to remove page numbering from abstract

Next-generation galaxy surveys will provide us with a wealth of high-precision cosmological data. To be able to use this information in an as unbiased manner as possible, theoretical accuracy must match the experimental precision. The three-point function, or bispectrum, is the first of the higher-order statistics beyond the power spectrum, and contains information both complementary and additional to what is contained in the power spectrum.

On large scales, the galaxy bispectrum will be a key probe for measuring primordial non-Gaussianity, improve constraints on cosmological parameters, and hence help discriminate between various models of inflation and other theories of the early universe. On these scales, a variety of relativistic effects come into play once the galaxy number-count fluctuation is projected onto the past light cone. It is important for these effects to be taken into account in our theoretical treatment of the bispectrum, as they will contaminate the primordial non-Gaussian signal and bias measurements. 

In this thesis, we review the relativistic projection effects in the galaxy bispectrum, and examine in detail how these relativistic effects contribute to the invariant multipoles of the galaxy bispectrum about the observer's line of sight. The Fourier-space bispectrum is complex, with an imaginary part arising from the relativistic effects only, which generates odd multipoles. This means that detection of this imaginary part is a smoking gun signal of the relativistic contributions, and we show that such a signal is in principle detectable in future surveys, although with a higher signal-to-noise ratio for spectroscopic surveys compared to 21cm intensity mapping surveys. Finally, we include local primordial non-Gaussianity in the theoretical description of the relativistic bispectrum, separating the relativistic corrections from the primordial signal, and use the bispectrum in Fisher matrix forecasts for cosmological parameters.