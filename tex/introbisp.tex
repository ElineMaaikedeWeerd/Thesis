\chapter{Bispectrum: overview}
\label{chapter:introbisp}


The galaxy bispectrum is the Fourier-space equivalent of the three-point function and is as such the first of the higher-order statistics beyond the power spectrum. Next-generation large scale structure such as Euclid (galaxy) and the SKA (21cm intensity mapping) will rely on a combination of the power spectrum and the bispectrum for high-precision measurements of primordial non-Gaussianity and for improvement of constraints on cosmological parameters. In particular, improvement on constraints on the primordial non-Gaussian parameter $\fnl$ will be crucial for discrimitation between various models of inflation and other theories of the early universe. 

\section{Relativistic effects}

Relativistic effects in the power spectrum, bispectrum?

Obsered galaxy number counts > how do relativistic effects come up in these, derivation of $\delta_g(z,\n)$ at both first and second order?

Some words about observable quantities and gauges maybe. God I need to read more 

