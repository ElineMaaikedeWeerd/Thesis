\chapter{Statistics of galaxy clustering}
\label{chapter:introbisp}

Desciption of the role of statistics in cosmology

The galaxy bispectrum is the Fourier-space equivalent of the three-point function and is as such the first of the higher-order statistics beyond the power spectrum. Next-generation large scale structure such as Euclid (galaxy) and the SKA (21cm intensity mapping) will rely on a combination of the power spectrum and the bispectrum for high-precision measurements of primordial non-Gaussianity and for improvement of constraints on cosmological parameters. In particular, improvement on constraints on the primordial non-Gaussian parameter $\fnl$ will be crucial for discrimitation between various models of inflation and other theories of the early universe. 

As it stands, our current understanding of the universe is that large scale structure of matter is a result of the growth of small primordial fluctuations which have functioned as seeds for structure growth in an otherwise homogeneous universe. These small fluctuations have been amplified by gravitatational instability, resulting in the formation of the structure that we know as the cosmic web on cosmological scales. Tests of theories describing these primordial fluctuations are statistical in nature for the following reasons. For one, there is no direct observationa access to primordial fluctuations, and additionally, the time-scales required to follow cosmological evolution of systems is much much longer than that over which observations are realistically possible. In essence this means that observations on the past lightcone how different objects at different phases of their evolution, and as a result tests of the evolution of large scale structure must be carried out statistically. 

A goal of theoretical cosmology is to make statistical predictions which depend on the statistical properties of primordial perturbations, which in turn lead to the formation of large scale structures in the universe. In these models, the observable universe is modelled simply as a stochastic realisation of a statistical ensemble of possibilities. The most widely considered models are based on the inflationary paradism, and generically give rise to adiabatic Gaussian initial fluctuations. In this case, the origin of stochasticity lies in quantum fluctuations that were generated in the early universe. 

\section{The two-point function}

A purely Gaussian field is fully described by the two-point correlation function or power spectrum. The two-point correlation function is defined as the joint ensemble average of the density at two different points $\x$ and $\x + \textbf{r}$, i.e. 
\begin{equation}
	\xi(r) = \langle \delta(\x) \delta(\x + \textbf{r}) \rangle,
\end{equation}
which is dependent on the distance $r$ between the two points only, due to statistical homogeneity and isotropy which are assumed throughout. Usually, the density contrast $\delta$ is expressed in terms of its Fourier space components, where our Fourier convention is 

\begin{equation}
	\delta({\x}) = \int \frac{\diff^3k}{(2\pi)^3} \, e^{\i \k \cdot \x } \delta(\x),
\end{equation}
where $\delta(\k)$ are complex random variables. Note that there are generally two Fourier conventions that are used in literature on the galaxy statistics, which lead to a difference of $(2\pi)^3$ in the definition of the power spectrum or two-point function. The other choice of Fourier convention is to reverse where the factor of $(2\pi)^3$ goes in the Fourier transforms, that is, using $f(\x) = \int \diff^3k e^{\i \k \cdot \x} f(\k) $ and $f(\k) = \int \frac{\diff^3 x}{(2\pi)^3} e^{-\i \k \cdot \x} f(\x)$ instead of our convention used here.

Since the density contrast is real, this means that we have
\begin{equation}
	\delta(k) = \delta^*(-\k).
\end{equation}

Similarly, the correlators can also be computed in Fourier space, as follows, 
\begin{equation}
	\langle \delta(\k) \delta(\k') \rangle = \int \diff^3 x \diff^3 r \langle \delta(\x) \delta(\x + \r)\rangle e^{-\i (\k + \k') \cdot \x - \i \k' \cdot \r}.
\end{equation}
This can be rewritten using the definition of the two-point correlation function as 
\begin{equation}
	\langle \delta(\k) \delta(\k') \rangle = \int \diff^3 x \diff^3 r \xi(r) e^{-\i (\k + \k') \cdot \x - \i \k' \cdot \r},
\end{equation}
and, performing one of the intregrals, 
\begin{align}
	\langle \delta(\k) \delta(\k') \rangle &= (2\pi)^3 \delta^D(\k + \k') \int \diff^3 r \xi(r) e^{\i \k \cdot \r} \\
	&\equiv (2\pi)^3 \delta^D(\k + \k') P(k),
\end{align}
where $P(k)$ is by definition the density power spectrum.

\section{The three-point function}

Higher-order correlation functions are defined as the connected part of the joint ensemble average of the density in an arbitrary number of locations. In principle it is possible define any order of correlation function like this, but they will rapidly become more computationally complex and expensive. In the case of a purely Gaussian field, the only non-vanishing connected part is the two-point correlation function. This is a direct consequence of Wick's theorem for Gaussian fields, and has a number of important consequences. Firstly 

Some blah blah about higher order statistics and their importance in future surveys (higher precision data, non-Gaussianities in the universe and effects that give rise to nonlinearities)

\subsection{The matter bispectrum}

Theoretical blah about the matter bispectrum, definitions, showing how various bispectrum signals translate to real space shapes and modulations of signal

\subsection{Matter bispectrum in observations}

Bispectrum in observation -- can skip over any CMB and just focus on lss

\subsection{Galaxy bias}

(Very) brief overview of the relevant bias


\subsection{Primordial non-Gaussianity}

Types of primodial non-Gaussianity in the bispectrum, their signatures on various scales, scale dependence it introduces in the clustering bias etc. 


cum sit enim proprium \\
viro sapienti \\
supra petram ponere \\
sedem fundamenti \\
stultus ego comparor \\
fluvio labenti \\
sub eodem tramite \\
nunquam permanenti 