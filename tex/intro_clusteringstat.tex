\chapter{Statistics of galaxy clustering}
\label{chapter:introbisp}

Desciption of the role of statistics in cosmology

The galaxy bispectrum is the Fourier-space equivalent of the three-point function and is as such the first of the higher-order statistics beyond the power spectrum. Next-generation large scale structure such as Euclid (galaxy) and the SKA (21cm intensity mapping) will rely on a combination of the power spectrum and the bispectrum for high-precision measurements of primordial non-Gaussianity and for improvement of constraints on cosmological parameters. In particular, improvement on constraints on the primordial non-Gaussian parameter $\fnl$ will be crucial for discrimitation between various models of inflation and other theories of the early universe. 

\section{The two-point function}

Brief introduction on the 2pt function/power spectrum including redshift space distortions, and its use in constraining cosmology

\section{The three-point function}

Some blah blah about higher order statistics and their importance in future surveys (higher precision data, non-Gaussianities in the universe and effects that give rise to nonlinearities)

\subsection{The matter bispectrum}

Theoretical blah about the matter bispectrum, definitions, showing how various bispectrum signals translate to real space shapes and modulations of signal

\subsection{Matter bispectrum in observations}

Bispectrum in observation -- can skip over any CMB and just focus on lss

\subsection{Galaxy bias}

(Very) brief overview of the relevant bias


\subsection{Primordial non-Gaussianity}

Types of primodial non-Gaussianity in the bispectrum, their signatures on various scales, scale dependence it introduces in the clustering bias etc. 