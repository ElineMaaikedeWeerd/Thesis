\chapter{Local primordial non-Gaussianity in the bispectrum}
\label{chapter:localpng}


Next-generation galaxy and 21cm intensity mapping surveys will rely on a combination of the power spectrum and bispectrum for high-precision measurements of primordial non-Gaussianity. In turn, these measurements will allow us to distinguish between various models of inflation. However, precision observations require theoretical precision at least at the same level. We extend the theoretical understanding of the galaxy bispectrum by incorporating a consistent general relativistic model of galaxy bias at second order, in the presence of local primordial non-Gaussianity. {The influence of primordial non-Gaussianity on the bispectrum extends beyond the  galaxy bias and the dark matter density, due to redshift-space effects. The standard redshift-space distortions at first and second order produce a well-known primordial non-Gaussian  imprint on the bispectrum. Relativistic corrections to redshift-space distortions
generate new contributions to this primordial non-Gaussian signal, arising from: (1)~a coupling of first-order scale-dependent bias with first-order relativistic observational effects, and (2)~linearly evolved non-Gaussianity in the second-order velocity and metric potentials which appear in relativistic observational effects.}
Our analysis allows for a consistent separation of the relativistic  `contamination' from the primordial signal, in order to avoid biasing the measurements by using an incorrect theoretical model. We show that the bias from using a Newtonian analysis of the squeezed bispectrum could be $\Delta \fnl\sim 5$ for a Stage IV H$\alpha$ survey.

