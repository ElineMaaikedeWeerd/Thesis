% !TEX root = ../thesis.tex

\chapter{The dipole of the galaxy bispectrum}
\label{chapter:dipole}

When carrying out large scale galaxy surveys, there is a variety of effects which distort the observed galaxy number counts, and for studies of galaxy clustering it is crucial for this to be accounted for properly. These effects arise from the fact that we observe on the past lightcone-- the galaxies' redshifts and the directions of the incoming photons all are distorted by inhomogeneities in our universe. In this chapter we examine the leading relativistic contributions in the galaxy bispectrum, which arise predominantly from RSD and other Doppler-type observational effects. These give rise to corrections at $\ord(\cH/k)$ in the galaxy bispectrum. Higher order $\cH/k$ contributions are, while being subdominant, still present in the galaxy bispectrum; and a full treatment of the bispectrum at all orders in $\ord(\cH/k)$ can be found from Chapter~\ref{chapter:ho} onwards. The content from this Chapter is taken from~\cite{Clarkson:2018dwn}.

\section{The dipole of the power spectrum}

The dominant RSD effect on galaxy number counts at first order is given by, 
\begin{equation}
    \delta_g(z,\k) = (b_{1}(z) + f (z) \mu^{2})\delta(z,\k)\,,
\end{equation}
where $\mu=\n\cdot\hat{\k}$, with $\n$ the line of sight direction, $f$ the growth rate, and $b_1$ is the linear bias. Henceforth the redshift dependence will be dropped for brevity as we are working at fixed redshift. At leading order, there is a Doppler type correction to this effect~\citep{Kaiser:1987qv,McDonald:2009dh,Challinor:2011bk} (see also \citep{Raccanelli:2016avd,Hall:2016bmm,Abramo:2017xnp}) proportional to $\bm{v}\cdot\n$, where $\bm{v}$ is the peculiar velocity:\footnote{\citep{Challinor:2011bk} provides the relativistic correction to the coefficient of $\bm{v}\cdot\n$ given in \citep{Kaiser:1987qv,McDonald:2009dh}.}
\begin{equation} \label{dopf}
\delta_g(\x) = b_{1} \delta(\x)-{\frac{1}{\cH}}\partial_r(\bm{v}\cdot\n) +{A}\,\bm{v}\cdot\n ~~\to~~
\end{equation}
\begin{equation} \delta_g(\k)=\Big(b_1+f\mu^2+\i {A}\,f \mu{\frac{\cH}{k}}\Big)\delta(\k)\,,\label{dopf2}
\end{equation}
where  
\begin{equation}
{A} = {b_{\textrm{e}}+{3}\Omega_m/2 -3 + (2-5s) (1-{1/ r\cH} )}\,.
\end{equation}
Here $b_{\textrm{e}}=\partial (a^3 \bar{n}_g)/\partial \ln a$ is the evolution of comoving galaxy number density, $s=-(2/5)\partial \ln \bar{n}_g/\partial \ln L$ is the magnification bias ($L$ is the threshold luminosity), $r$ is the comoving radial distance ($\partial_r=\bm n\cdot\bm\nabla$) and we have assumed a $\Lambda$CDM background ($\cH'/\cH^2=1-3\Omega_m/2$, where $\cH$ is the conformal Hubble rate, a prime is differentiation with respect to conformal time, $\Omega_m$ is the evolving density contrast). In the  Fourier space expression~\eqref{dopf2} we can read off the relative contribution of each term by how they scale with $k$: terms like $\cH/k$ are suppressed on small scales when $\cH/k\ll1$ but become important around and above the equality scale. 

Although the galaxy density contrast~\eqref{dopf2} is complex, the power spectrum of a single tracer is real:
\begin{equation}
\big\langle \delta_g(\k)\delta_g(-\k)\big\rangle
=\Big[ \big(b_{1} + f\mu^{2}\big)^2+\Big(A\,f \mu \frac{\cH}{k}\Big)^2\Big] \big\langle \delta(\k)\delta(-\k)\big\rangle \,,\nonumber
\end{equation}
since $\mu_{-\k}=-\mu_{\k}$ enforces a cancellation of the imaginary part, and the RSD contribution is separate from the Doppler term.
However, if we consider the cross-power spectrum for {\em two} matter tracers, this cancellation breaks down, and  there is an imaginary part in the cross-power~\citep{McDonald:2009dh,Bonvin:2014owa},
\begin{align}
P_{g \tilde g}(k)&= \Big\{\Big[ \big(b_{1} + f\mu^{2}\big)\big(\tilde b_{1} + f\mu^{2}\big)+A\tilde A f^2\mu^2 {\frac{\cH^2}{k^2}}\Big]\nonumber\\
&
+\i f\mu\Big[\big(\tilde b_{1} + f\mu^{2}\big)A-\big(b_{1} + f\mu^{2}\big)\tilde A\Big]{\frac{\cH}{k}} \Big\}P(k) \,.\nonumber
\end{align}
While the Doppler contribution to $P_g$ is $\ord((\cH/k)^{2})$,  the Doppler contribution to $P_{g\tilde g}$ mixes with the density and RSD to give an additional less suppressed part, i.e. $\ord(\cH/k)$. The nonzero multipoles of $P_g$ are $\ell=0,2,4$, whereas  $P_{g \tilde g}$ has a nonzero dipole (as well as a smaller octupole).  There are also further relativistic corrections to this dipole part of the cross power spectrum~\citep{DiDio:2018zmk}.


\section{Leading order relativistic contributions to the bispectrum}

A natural question is: what about the galaxy bispectrum? In the standard `Newtonian' approximation, with only RSD, the galaxy bispectrum for a single tracer at fixed redshift has no dipole, and only has even multipoles~\citep{Scoccimarro:1999ed,Nan:2017oaq}. But with a lightcone corrected galaxy density contrast, the 3-point correlator, even for a {\em single} tracer, will no longer be an even function of $\k_a\!\cdot\n \,(a=1,2,3)$. In order to compute the consequent contribution to the galaxy bispectrum, \eqref{dopf} is not sufficient: we need its second-order generalisation, $\delta_g \to \delta_g+\delta^{(2)}_g/2$.

At second order, the Doppler correction in~\eqref{dopf} generalises to $A\, \bm{v}^{(2)}\!\cdot\n$, but there are also quadratic coupling terms. The couplings involve not only the Doppler effect, but also radial gradients of the potential (`gravitational redshift'), volume distortion effects, and second-order corrections to the density contrast. Most of these contributions are small, but those that scale as $(\cH/k)\delta^2$ are not, even on equality scales. Except on super-equality scales we can often neglect any terms $\ord((\cH/k)^{2})$ and higher, which makes the calculation considerably simpler. The treatment of higher-order terms is left to chapter~\ref{chapter:multipoles}.

The leading correction can be extracted from the general expressions that include all relativistic  corrections to the Newtonian approximation, as given in~\cite{Bertacca:2014hwa} (see also~\cite{Bertacca:2014dra,Yoo:2014sfa,DiDio:2014lka,Jolicoeur:2017nyt,DiDio:2018zmk}), 
\begin{align}
\delta^{(2)}_{g{\mathrm{D}}} &= A \, \bm{v}^{(2)}\!\!\cdot\n+2{C}(\bm{v}\cdot\n)\delta +2 {\frac{E}{\cH}}(\bm{v}\cdot\n)\partial_r(\bm{v}\cdot\n)\label{dg2}
\\ & \nonumber
+ 2 \frac{b_1}{\cH}\phi\, \partial_r\delta
+ \frac{2}{\cH^2} \big[\bm{v}\cdot\n \,\partial_r^2\phi-\phi\, \partial_r^2 (\bm{v}\cdot\n) \big] - \frac{2}{\cH}\partial_r (\bm{v}\cdot\bm{v}), 
\end{align}
where $\phi$ is the gravitational potential, and the coefficients C and E are,
\begin{equation}
    C = b_1(A + f)+{b_1'/ \cH}+ 2(1-{1/ r\cH} ){\partial b_1/ \partial\ln L}\,,
\end{equation}
and
\begin{equation}
E = {4-2A-{\frac{3}{2}}\Omega_m}\,.
\end{equation}
This is in agreement with the independent re-derivation of the leading correction given in~\cite{DiDio:2018zmk}. We have corrected a typo in the last bracket of line 1 of Eq. (2.15): $-f_{\mathrm{evo}}\to -2f_{\mathrm{evo}}\equiv -2b_{\mathrm{e}}$. Note that our $\n$ is minus theirs, and they use the convention $\delta_g+\delta^{(2)}_g$.
All but one of the contributions to this leading term contain Doppler contributions, so we label these terms with a D subscript. In this sense, they can be thought of as the relativistic correction to redshift space distortions, but their origin is considerably more subtle than in the Newtonian picture~\citep{Bertacca:2014dra,DiDio:2018zmk}. These relativistic corrections all arise as projections along the line of sight $\n$. It is this projection that is responsible for the dipole in the observed bispectrum. Beyond these leading terms in~\eqref{dg2} there are a host of local coupled terms which appear on larger scales. 

We follow most work on the Fourier bispectrum and neglect the effect of lensing magnification. This is reasonable for correlations at the same redshift and when using very thin redshift bins allowed by spectroscopic surveys~\citep{DiDio:2018unb}. We also use the standard plane-parallel approximation, which is reasonable at high redshift. However, we note that wide-angle effects in the power spectrum can be of the same order of magnitude as the Doppler-type effects in certain circumstances~\citep{Tansella:2017rpi}, and these should be incorporated in a more complete treatment.

The galaxy bispectrum is defined in Fourier space by,
\begin{align}
B_{g}( \mathbf{k}_{1},  \mathbf{k}_{2},  \k_3) &= { \mathcal{K}( \k_{1})\mathcal{K}({\k}_{2}) \mathcal{K}^{(2)}(  \mathbf{k}_{1},  \mathbf{k}_{2}, \k_3)}
P(k_{1})P(k_{2}) \nonumber\\ 
&+\text{2 cyclic permutations}\,.\label{eq:fsbispdef}
\end{align}
The first-order kernel $\mathcal{K}=\mathcal{K}_{\mathrm{N}}+\mathcal{K}_{\mathrm{D}}$ is given by the term in brackets in~\eqref{dopf2}.
At second order, $\mathcal{K}^{(2)}=\mathcal{K}_{\mathrm{N}}^{(2)}+\mathcal{K}_{\mathrm{D}}^{(2)}$, where
 the Newtonian kernel is~\citep{Verde:1998zr}
\begin{align}
\mathcal{K}_{\mathrm{N}}^{(2)} (\ka,\kb,\kc) &= b_{2} + b_{1}F_{2}(\ka,\kb) + b_s S_{2}(\ka,\kb) \nonumber \\
& + f \mu_3^{2} G_{2}(\ka,\kb)
+ {\cal Z}_2(\ka,\kb)\,.   \label{kn2}  
\end{align}
Here $F_2(\ka,\kb)$ is the standard Newtonian mode-coupling kernel for $\Lambda$CDM~\citep{Villa:2015ppa}, 
\begin{equation}
    F_2(a,\ka,\kb) = 1 + \frac{F(a)}{D(a)^2} + \left( \frac{k_1}{k_2} + \frac{k_2}{k_1} \right) \hat{\k}_1 \cdot \hat{\k}_2 + \left[ 1 - \frac{F(a)}{D(a)^2} \right]\left( \hat{\k}_1 \cdot \hat{k}_2 \right)^2\,,
\end{equation}
where $F$ is the second-order growth factor. For $\Lambda$CDM, the Einstein-De Sitter relation $F/D^2 = 3/7$ is a very good approximation. Using this approximation, $F_2$ is essentially time-independent. $G_2({\k_{1}},  {\k_{2}},\kc)$ is the second-order velocity kernel, 
\begin{equation}
    G_2(a,\ka,\kb) = \frac{F'(a)}{D(a) D'(a)} + \left( \frac{k_1}{k_2} + \frac{k_2}{k_1} \right) \hat{\k}_1 \cdot \hat{\k}_2 + \left( 2 - \frac{F'(a)}{D(a) D'(a)} \right) \left( \hat{\k}_1 \cdot \hat{\k}_2 \right)^2\,,
\end{equation}
where, since we use the Einstein-De Sitter approximation $F/D^2 = 3/7$ in $F_2$, we have $F'/(D D') = 6/7$ in $G_2$. We use a local bias model~\citep{Desjacques:2016bnm}, which includes tidal bias kernel
\begin{equation}
    S_2(\ka,\kb) = (\hat{\k}_1 \cdot \hat{\k}_2)^2 - \frac{1}{3}\,, 
\end{equation}
with tidal bias $b_s$. Finally, the kernel $\mathcal{Z}_2(\ka,\kb)$ incorporates the non-linear Kaiser RSD contribution~\citep{Verde:1998zr,Scoccimarro:1999ed}, 
\begin{equation}
    \mathcal{Z}_2(\ka,\kb) = b_1 f (\mu_1 k_1 + \mu_2 k_2) \left( \frac{\mu_1}{k_1} + \frac{\mu_2}{k_2} \right) + f^2 \frac{\mu_1 \mu_2}{k_1 k_2} \left( \mu_1 k_1 + \mu_2 k_2 \right)^2\,.
\end{equation}

The Doppler correction to \eqref{kn2} in Fourier space follows from~\eqref{dg2}~\citep{Jolicoeur:2017eyi},
\begin{align}
\mathcal{K}^\tw_{\mathrm{D}}(\ka,\kb,\kc) &= \i \cH
\left[ - \frac{3}{2} \left( \mu_{1} \frac{k_{1}}{k_2^2} + \mu_{2} \frac{k_{2}}{k_1^2} \right) \Omega_m b_1 + 2\mu_{12} \left( \frac{\mu_{1}}{k_2} + \frac{\mu_{2}}{k_1}\right) f^2  \right. \nonumber \\
& \left. \left( \frac{\mu_{1}}{k_1} + \frac{\mu_{2}}{k_2} \right){C} f - \frac{3}{2} \left( \mu_{1}^3 \frac{k_{1}}{k_2^2} + \mu_{2}^3 \frac{k_{2}}{k_1^2} \right)\Omega_m f \nonumber\right.  \\
&\left. + \mu_{1}\mu_{2}\left( \frac{\mu_{1}}{k_2} + \frac{\mu_{2}}{k_1} \right) {\Big( \frac{3}{2} \Omega_m - E f\Big) f} + \frac{\mu_{3}}{k_{3}} G_{2}(\bm{k}_{1}, \bm{k}_{2},\kc){A} f\right]\,, \label{d2}
\end{align}
where $\mu_{ab}=\hat{\k}_a\cdot \hat{\k}_b$ and $\mu_a=\hat{\k}_a\cdot\bm n$. 
The Newtonian kernel~\eqref{kn2} scales as $(\cH/k)^{0}$, while the Doppler kernel~\eqref{d2} scales as $(\cH/k)$. 
Using~\eqref{kn2} and~\eqref{d2} in~\eqref{eq:fsbispdef}, and dropping terms that scale as $(\cH/k)^{2}$ and $(\cH/k)^{3}$,
we find that
\begin{align}
B_{g{\mathrm{N}}}(\ka,\kb,\kc) &=  \ko_{\mathrm{N}}(\ka)\ko_{\mathrm{N}}(\kb)\kt_{\mathrm{N}}(\ka,\kb,\kc)\,P(k_{1})P(k_{2})  \nonumber \\
& +\text{2 cyclic permutations,} \\
B_{g{\mathrm{D}}}(\ka,\kb,\kc) &= \left\{
\ko_{\mathrm{N}}(\ka)\ko_{\mathrm{N}}(\kb)\kt_{\mathrm{D}}(\ka,\kb,\kc) \right.
\nonumber\\ \label{bd}
& \left. +\left[ \ko_{\mathrm{N}}(\ka)\ko_{\mathrm{D}}(\kb)+\ko_{\mathrm{D}}(\ka)\ko_{\mathrm{N}}(\kb) \right]\kt_{\mathrm{N}}(\ka,\kb,\kc)
\right\}
P(k_{1})P(k_{2}) \nonumber \\
& +\text{2 cyclic permutations.}
\end{align}
Since \eqref{d2} scales as $\cH/k$ it is purely imaginary, as all these contributions have at least one $\k$ projected along the line of sight~-- i.e.,  they contain odd powers of $\mu_a$'s. This means that {\em the leading relativistic correction in the observed galaxy Fourier bispectrum of a single tracer is a purely imaginary addition to the Newtonian approximation.} On larger scales, terms $\ord((\cH/k)^{2})$ and higher appear in both the real and imaginary parts, with the kernels given in~\cite{Umeh:2016nuh,Jolicoeur:2017nyt,Jolicoeur:2017eyi,Jolicoeur:2018blf}. We include the higher-order terms in our plots below, and a full treatment can be found in chapter~\ref{chapter:multipoles}.

\section{Extracting the dipole}

The bispectrum can be considered as a function of
$k_1,k_2,k_3,\mu_1, \mu_2,\mu_3$ and $\varphi$, which is the azimuthal angle giving the orientation of the triangle  relative to $\n$. In order to extract the dipole it is easiest to write $\mu_3=-(k_1\mu_1+k_2\mu_2)/k_3$, so that we can write $B_g = \sum_{i,j} \mathcal{B}_{ij}(\i\mu_1)^i(\i\mu_2)^j$, where $i,j=0\ldots6$ which factors out the angular dependence multiplying real coefficients $\mathcal{B}_{ij}$ with no angular dependence. Then,
 use the identity 
$%\be
\mu_2=\mu_1\cos\theta+\sqrt{1-\mu_1^2}\,\sin\theta\,\cos\varphi\,,
$ %\ee 
where $\theta=\theta_{12}$ (and we define $\mu=\cos\theta$~-- note that $\theta$ is the angle outside the triangle as the $\k_a$'s are head-to-tail). We use standard orthonormal spherical harmonics with the triangle lying in the $y-z$ plane, with $\bm k_1$ aligned along the $z$-axis~\citep{Nan:2017oaq}, see Figure~\ref{fig:geometry_overview1} for a schematic overview of the relevant angles and vectors of our decomposition.
\begin{figure}[ht]
    \centering
    \includegraphics[width=0.6\linewidth]{fig/fig.pdf}
    \caption{Overview of the relevant vectors and angles for the Fourier-space bispectrum. \label{fig:geometry_overview1} }
\end{figure}       
Then we have $Y_{\ell m}(\mu_1,\varphi)$, so that we can write $B_g =\sum_{\ell m} B_{\ell m}Y_{\ell m}(\mu_1,\varphi)$. The leading relativistic terms we consider here generate odd-power multipoles up to $\ell=7$, and the full expression generates even and odd multipoles up to $\ell=8$-- see Chapter~\ref{chapter:multipoles}. 
Different powers of $(\i\mu_1)$ and $(\i\mu_2)$ contribute to the dipole as follows,
\begin{align}
\arraycolsep=1.0pt\def\arraystretch{1}
\int\mathrm{d}\Omega (\i\mu_1)^a  (\i \mu_2)^b Y^*_{1 m} & = 
 \delta_{m,0}\frac{\i\sqrt{3\pi}}{15}\left[ \begin {array}{ccccc} 
 0&10\mu&0&-6\mu  &\\ 
10&0&-4{\mu}^{2}-2&0& \cdots\\ 
0&-6\,\mu&0&{\frac {12{
\mu}^{3}+18\mu}{7}}& \\ 
-6&0&{\frac {24
\,{\mu}^{2}+6}{7}}&0& \\
 &\vdots & & & \ddots
\end {array} \right] \nonumber \\
& +
\delta_{m,\pm1}\frac{\sqrt{6\pi}}{15}
 \left[ \begin {array}{ccccc} 0&-5&0&3&\\ 
 0&0&2\,\mu&0
& \cdots\\ 
0&1&0&-\frac{6{\mu}^{2}+3}{7}&\\ 
0&0
&-\frac{6}{7}\,\mu&0&\\
 &\vdots & & & \ddots
\end {array} \right] \sin\theta\,,
\label{dkjsncdjcnsk}
\end{align}
where each matrix element corresponds to a particular combination of $a,b$,
where the matrix indices run over the values $a=0\ldots6, b=0\ldots6$, with powers above 3 not written above; these are polynomials in $\mu$ up to order 6. From this we can read off the terms from $\mathcal{K}_\text{D}$ contribute to differing $m=0,\pm1$. In particular, if $i+j$ is even~-- i.e., the real part of the bispectrum~--  there is no contribution: only the imaginary terms, corresponding to $i+j$ odd, contribute. For the monopole, only $i+j$ even contribute. Therefore, at $\ord(\cH/k)$, \emph{the monopole of the bispectrum is the Newtonian part, while the dipole is purely from the relativistic corrections.  The presence of the dipole is therefore a `smoking gun' signal for the leading relativistic correction to the bispectrum.} At order $\ord((\cH/k)^{2})$, relativistic terms appear in the monopole, which were considered in \cite{Umeh:2016nuh,Jolicoeur:2017nyt,Jolicoeur:2017eyi,Jolicoeur:2018blf}.\\


\section{Squeezed, equilateral and flattened limits}

It is relatively straightforward to understand the type of dipole generated in different triangular configurations in our conventions. In particular, for the $\ord(\cH/k)$ relativistic dipole:
\begin{itemize}
\item The squeezed case is zero for $m=0$, and is non-zero for $m=\pm1$. We see this directly from \eqref{dkjsncdjcnsk}: with $\mu=-1$ the $m=0$ contribution is anti-symmetric in $i,j$ while $\mathcal{B}_{ij}$ is symmetric in this limit.
\item In the equilateral case, the dipole is zero (this is the case for all orders in $\cH/k$).
\item The flattened case ($k_1=k_2=k_3/2,\theta=0$) is zero for $m=\pm1$ (for all orders in $\cH/k$), but is non-zero for $m=0$. This can be seen directly from \eqref{dkjsncdjcnsk} with $\theta=0$.
\end{itemize}
%Geometrically, we can understand these by rotating the triangle about its centre with respect to $\bm n$. For the equilateral case, the symmetry of the triangle means that the dipole part cancels out. For the flattened case, where there is a vertex at the centre, an anti-symmetry occurs only parallel to $\bm n$, and cancels in the plane orthogonal. For the squeezed case, this is anti-symmetric both along and perpendicular to $\bm n$, so excites both $m=\pm1$. 
 
To show the equilateral case is zero is a lengthy calculation involving many cancellations.  Let us illustrate instead the squeezed case. We write $k_1=k_2=\sqrt{1+\varepsilon^2}k_S, k_3=2\ep k_S$.
In this case the triangle has small angle $2\ep$ and equal angles $\pi/2-\ep$, where the squeezed limit is $\ep\to0$. It is convenient to replace $(1,2,3)$ by $(S,-S,L)$.
Then to $O(\ep)$,
$%\bea
k_{-S}= k_{S}\,,~~k_L=2\ep k_S\,,~~  
\mu_{-S}=-\mu_S-2\ep\mu_L\,,
%\nonumber\\&&
\mu_L = -\sqrt{1-\mu_S^2}\,\cos\varphi - {\ep\mu_S}\,.
%\hat{\k}_2=-\hat{\k}_1-2\ep\,\hat{\k}_3 \\ &&
$% \label{mu}
%\eea
 ~In this limit, the permutations of the relativistic kernels become
\begin{align}
% {\cal K}^{(2)}_{\mathrm{D}}(\k_S,\k_{-S},\k_L) &=&0\\
& {\cal K}^{(2)}_{\mathrm{D}}(\k_L,\k_S,\k_{-S}) = \i {\cH}
{\bigg[- \frac{3}{2}\Omega_m b_1\mu_S \frac{k_S}{k_L^2}+ C f \frac{\mu_L}{k_L} }
\nonumber\\
& - \frac{3}{2} \Omega_m f\mu_S^3 \frac{k_S}{k_L^2} +\Big( \frac{3}{2} \Omega_m- Ef \Big) f\mu_S^2 \frac{\mu_L}{k_L}\bigg] 
\label{k2d1} 
% {\cal K}^{(2)}_{\mathrm{D}}(\k_{-S},\k_L,\k_S) &=&  {\cal K}^{(2)}_{\mathrm{D}}(\k_L,\k_S,\k_{-S})\Big|_{\mu_S\to \roy{\mu_{-S}}}  \label{k2d2}
\end{align}
and $ {\cal K}^{(2)}_{\mathrm{D}}(\k_{-S},\k_L,\k_S) =  {\cal K}^{(2)}_{\mathrm{D}}(\k_L,\k_S,\k_{-S})\big|_{\mu_S\to {\mu_{-S}}}$ while ${\cal K}^{(2)}_{\mathrm{D}}(\k_S,\k_{-S},\k_L) =0$. 
In the squeezed limit of the cyclic sum~\eqref{eq:fsbispdef}, the terms $\cK^{(2)}(\k_L,\k_S,\k_{-S})$ and $\cK^{(2)}(\k_{-S},\k_L,\k_S)$ appear only in the form ${\cal K}^{(2)}(\k_L,\k_S,\k_{-S})+{\cal K}^{(2)}(\k_{-S},\k_L,\k_S)$. This sum regularises the divergent $k_S/k_L=(2\ep)^{-1}$ and $k_S/k_L^2=(2\ep k_L)^{-1}$ terms.  We obtain the bispectrum in the squeezed limit,
\begin{align}
&B_{g}^{\mathrm{sq}} = b_{1S}b_{1L}b_{SL}\, P_LP_S+\i b_{1S}\Big\{b_{SL}f A + {\frac{3}{2}} \Omega_m b_{1S}b_{1L} 
\nonumber\\
&+2b_{1L}f C+ b_{1L}\mu_S^2\Big[{\frac{3}{2}}\Omega_m- Ef \Big] \Big\}P_LP_S\,\mu_L \frac{\cH}{k_L} \,,\label{bgd}
\end{align}
where $P_{S,L}=P(k_{S,L})$, $b_{1S,L}\equiv  b_1+f\mu_{S,L}^2$ and
\begin{align}
% \,, \label{b1sl}~~
b_{SL} \equiv 2b_2+ \frac{43}{21}b_1 - \frac{4}{21}
%\nonumber\\&&
+\Big(2b_1+ \frac{5}{7}\Big)f\mu_S^2%+b_1f\mu_L^2+{f^2}\mu_S^2\mu_L^2 
+f\mu_L^2 b_{1S}
\,. \nonumber\label{bsl}
\end{align}
%Note that we can neglect the $P(k_S)^2$ term relative to the $P(k_L)P(k_S)$ term in $B_{g {\mathrm{N}}}^{\rm sq}$.
Note that only the first term in the squeezed bispectrum comes from the Newtonian limit. 

The type of dipole extracted from this term is seen as follows. To this order we can write $\mu_{S}^2=\mu_{S}\mu_{-S}$. Then,  since $\mu_L=-2(\mu_S+\mu_{-S})/
\varepsilon$, we see that the $m=0$ term is zero because $B_{g {\mathrm{D}}}^{\mathrm{sq}}$ is symmetric in $\mu_{S}^i\mu_{-S}^j$ under $i\leftrightarrow j$, while the $m=0$ term is antisymmetric in~\eqref{dkjsncdjcnsk}. This leaves just the $m=\pm1$ contribution in~\eqref{dkjsncdjcnsk}.

\begin{figure}[!ht]
\begin{center}
%\includegraphics[width=\columnwidth]{flattened-squeezed-02}
\includegraphics[width=\columnwidth]{fig/figuresv2-02}
\caption{ The absolute value of the bispectrum dipole at $z=1$  as a function of triangle size, in the flattened (Left, $\theta=2^\circ$, for intensity mapping bias) and squeezed (Right, $\theta=178^\circ$, for Euclid-like bias) configurations, with $k_3$ as the horizontal axis. Red is the $m=0$ part and blue is $m=\pm1$. Dashed (and dotted) lines show up to the $\ord(\cH/k)$ terms considered analytically here, while solid lines indicate larger-scale contributions. For reference the monopole is in black, with the dotted line the Newtonian part.  (The zero-crossing in the monopole for the squeezed case is a result of the tidal bias.)}
\label{snakcjnsdlkcans}
\end{center}
\end{figure}
\begin{figure}[!ht]
\begin{center}
%\includegraphics[width=\columnwidth]{shapes-01}
\includegraphics[width=\columnwidth]{fig/figuresv2-01}
\caption{ (Left) We show the dipoles as a function of $\theta$ with a bias appropriate for a Euclid-like survey, for $k_1=k_2=0.01$\,Mpc$^{-1}$. The left of the plot corresponds to the flattened case where the $m=0$ (red) dipole reaches 10\% of the monopole.  (Right) We show the IM signal with $k_1=k_2=0.1$\,Mpc$^{-1}$ versus the long mode $k_3$. Except for very long modes $\theta\approx\pi$, our $\ord(\cH/k)$ truncation is a very good approximation in these examples. }
\label{sankcjnakjdcs}
\end{center}
\end{figure}
 
\section{The dipole in intensity mapping and galaxy surveys}

We now consider the amplitude of the dipole relevant for upcoming galaxy surveys, which have different bias parameters. We consider two different types of survey: an SKA intensity mapping of 21\,cm radio emission, as well as a Euclid-like optical/infrared spectroscopic survey.
An intensity map of the 21cm emission of neutral hydrogen (HI) in the post-reionization Universe records the total emission in galaxies containing HI, without detecting individual galaxies. There is an equivalence between the brightness temperature contrast and number count contrast~\cite{Umeh:2015gza}. For IM we use the bias parameters at $z=1$, 
$b_1 = 0.856, b_2 = -0.321, b_1' = -0.5\times10^{-4}, b_e = -0.5, b_e'=0, s = 2/5$~\cite{Fonseca:2018hsu,Umeh:2015gza}
while for the spectroscopic survey we use 
$ b_1 = 1.3,b_2 = -0.74, b_1' = -1.6\times10^{-4},  b_e = -4, b_e' = 0, s = -0.95$~\cite{Camera:2018jys,Yankelevich:2018uaz}.
For intensity mapping, $ \partial b_1/\partial \ln L =0$ and we assume it is zero for simplicity for the spectroscopic survey. We use a LCDM model with standard parameters $\Omega_m=0.314, h=0.67, f_\text{baryon}=0.157, n_s=0.968$. Plots are presented using linear power spectra generated using CAMB~\cite{Lewis:1999bs}.

In Fig.~\eqref{snakcjnsdlkcans}, we show how changing the scale of a fixed triangle changes the amplitude of the dipole, with reference to the monopole. In the flattened case with $m=0$ we see the signal peaks for triangles below the equality scale, while for squeezed shapes, with $m=\pm1$, the signal is smaller, and peaks when the long mode approaches the Hubble scale. 
In Fig.~\eqref{sankcjnakjdcs}, we change the shape with fixed $k_1=k_2$ for both galaxy and IM surveys. We confirm our analytical results that the equilateral limit is zero, as well as the other limits. For triangles between right-angle and flattened the dipole is more than 10\% of the monopole, and the signal is largest in the flattened case~-- except in the extreme squeezed limit (not shown). 

\section{Conclusions}

We have shown for the first time that the relativistic galaxy bispectrum has a leading correction which is a local dipole with respect to the observers line of sight. In contrast to the power spectrum, this dipole exists even for a single tracer. We have shown analytically how the dipole is generated for the leading terms, and numerically we have included all local contributions, which show up above the equality scale. We have neglected integrated terms which will also contribute to the dipole, but their inclusion in a Fourier space bispectrum is non-trivial. Local relativistic corrections will induce all multipoles up to $\ell=8$ at every $m$, in contrast to the Newtonian case which only induces even $\ell=0,2,4$. We will investigate these new multipoles in a forthcoming publication. 

We have shown that this dipole is large with respect to the monopole in both the flattened and squeezed limits, which excite different orders of the dipole orientation $m$.  We have shown that even on equality scales it is about 10\% of the monopole at $z=1$ for flattened shapes which have the largest amplitude. In more squeezed cases where the short mode is $\sim10$\,Mpc the dipole can also be a large part of the IM signal. Furthermore, although we have only considered Gaussian initial conditions here, the dipole will be unaffected by non-Gaussianity at leading order because these corrections start at $\ord((\cH/k)^2)$, making our predictions relatively robust to this. This implies that the dipole of the bispectrum is a unique signature of general relativity on cosmological scales, and therefore offers a new observational window onto modifications of general relativity. 




